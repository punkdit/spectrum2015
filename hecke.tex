
%\documentclass[12pt,notitlepage,aps,pra,longbibliography,nofootinbib,tightenlines]{revtex4}
%\documentclass[12pt,notitlepage,longbibliography,nofootinbib,tightenlines]{revtex4-1}

\documentclass[12pt]{article}

%\documentclass[12pt,a4]{revtex4}
%\documentclass[12pt]{article}
%\documentclass[11pt, twocolumn]{article}

%\usepackage{epsf}
\usepackage{amsmath}
\usepackage{color}
\usepackage{natbib}
%\usepackage{cite}
\usepackage{fullpage} % uses 20 percent less pages.

\usepackage{framed}

\RequirePackage{amsmath}
\RequirePackage{amssymb}
\RequirePackage{amsthm}
%\RequirePackage{algorithmic}
%\RequirePackage{algorithm}
%\RequirePackage{theorem}
%\RequirePackage{eucal}
\RequirePackage{color}
\RequirePackage{xcolor}
\RequirePackage{url}
\RequirePackage{mdwlist}

\RequirePackage[all]{xy}
\CompileMatrices
\RequirePackage{hyperref}
\RequirePackage{graphicx}
%\RequirePackage[dvips]{geometry}

\newtheorem{theorem}{Theorem}
\newtheorem{lemma}{Lemma}

%\renewenvironment{framed}[1][\hsize]{%
%\def\FrameCommand{{\color{red}\vrule width 3pt}\hspace{0pt}\fboxsep=\FrameSep\colorbox{yellow}}%
%\MakeFramed{\hsize#1\advance\hsize-\width\FrameRestore}}
%{\endMakeFramed}

%\renewenvironment{framed}[1][\hsize]{%
%\def\FrameCommand{{\color{red}\vrule width 3pt}\hspace{0pt}\fboxsep=\FrameSep\colorbox{yellow}}%
%\MakeFramed{\hsize0.8\linewidth\advance\hsize-\width\FrameRestore}}
%{\endMakeFramed}

\renewenvironment{framed}[1][\hsize]{%
\def\FrameCommand{{\color{black}\vrule width 3pt}\hspace{0pt}\fboxsep=\FrameSep\colorbox{lightgray}}%
\MakeFramed{\hsize0.8\linewidth\advance\hsize-\width\FrameRestore}}
{\endMakeFramed}


\begin{document}

\title{Notes on Hecke Algebras}

\author{Simon Burton}
%\affiliation{Centre for Engineered Quantum Systems, School of Physics, The University of Sydney}

\date{\today}

%\begin{abstract}
%\end{abstract}

\maketitle

%\begin{abstract}
%\end{abstract}

%\newpage
%\tableofcontents
%\newpage

% CUT HERE

\def\Complex{\mathbb{C}}
\def\C{\mathbb{C}}
\def\R{\mathbb{R}}
\def\Z{\mathbb{Z}}
%\def\Ham{\mathcal{H}} % meh..
\def\Ham{H} 
\def\Pauli{\mathcal{P}}
\def\Spec{\mbox{Spec}}
\def\Proveit{{\it (Proof??)}}
\def\GL{\mathrm{GL}}
\def\half{\frac{1}{2}}
\def\Stab{S}
\def\Field{\mathcal{F}}
\def\Defn#1{\underline{#1}}

\newcommand{\ket}[1]{|{#1}\rangle}
\newcommand{\expect}[1]{\langle{#1}\rangle}
\newcommand{\bra}[1]{\langle{#1}|}
\newcommand{\ketbra}[2]{\ket{#1}\!\bra{#2}}
\newcommand{\braket}[2]{\langle{#1}|{#2}\rangle}


%%%%%%%%%%%%%%%%%%%%%%%%%%%%%%%%%%%%%%%%%%%%%%%%%%%%%%%%%%%%%%%%%%%%%%%%%%%%%%%
%
%%%%%%%%%%%%%%%%%%%%%%%%%%%%%%%%%%%%%%%%%%%%%%%%%%%%%%%%%%%%%%%%%%%%%%%%%%%%%%%
%

%\section{Representations}
%
%\subsection{Motivating examples}

A \Defn{Coxeter group}
has generators $\{s_1, .. s_n \}$
with relations
\begin{align*}
    s_i^2 &= 1 \ \ \mbox{for}\ i=1,..,n\\
    (s_i s_j)^{m(i,j)} &= 1\ \ \mbox{for}\ i,j=1,..,n.
\end{align*}
where $m(i,j)\in\{2,3,4,6\}.$
See \cite{Garrett1997}, \cite{Baez2010}.
A \Defn{dihedral group}
is a Coxeter group with two generators.

The \Defn{Dynkin diagram} (or Coxeter diagram)
associated to a Coxeter group is an undirected graph
with vertices $1,..,n$ and edges $(i,j)$ when $m(i,j)\ge 3.$
The edges are labelled with the value $m(i,j)$.

The \Defn{Hecke algebra} associated with a Dynkin diagram $D$
and $q\ne 0$ 
is the associative algebra $H(D,q)$ with
generators $\sigma_i,\ i=1,..,n$ and relations
\begin{align*}
\sigma_i^2 &= (q-1)\sigma_i + q \\
\sigma_i \sigma_j &= \sigma_j \sigma_i \ \mbox{if}\ m(i,j)=2 \\
\sigma_i \sigma_j \sigma_i &= \sigma_j \sigma_i \sigma_j \ \mbox{if}\ m(i,j)=3 \\
(\sigma_i \sigma_j)^2 &= (\sigma_j \sigma_i)^2 \ \mbox{if}\ m(i,j)=4 \\
(\sigma_i \sigma_j)^3 &= (\sigma_j \sigma_i)^3 \ \mbox{if}\ m(i,j)=6.
\end{align*}
Note that for $m(i,j)=3$ this gives a Yang-Baxter equation and this
explains the connection to knot-theory in such cases (these are the $A_k$ Dynkin diagrams.)

We can factorize the first relation as
$$
    (\sigma_i - q)(\sigma_i + 1) = 0, \ \ i=1,..,n.
$$

The $n$-qubit (real) \Defn{Pauli group} is defined with
$2n$ generators $\{X_i, Z_i\}_{i=1,..,n}$
and relations
\begin{align*}
    X_i^2 = 1,       \ Z_i^2 = 1,\ \ i=1,..,n\\
    (X_i X_j)^2 = 1, \ (Z_i Z_j)^2 = 1,\ \ i,j=1,..,n\\
    (X_i Z_j)^2 = 1, \ (X_i Z_i)^4 = 1, \ i\ne j,\ i,j=1,..,n.
\end{align*}
It follows that this group is an example of a Coxeter group,
but building a Hecke algebra as above will not yield a 
Yang-Baxter equation.


Given a group $G$ and subgroups $H, K$ a
\Defn{double coset}
is the set $HgK=\{hgk | h\in H, k\in K\}.$
The set of double cosets is written $H\backslash G/K.$

%We have an equivelance relation on $G$ given by
%$$
%    g \sim g' \mbox{if} 
%$$
%
%Each double coset $HgK$ is an equivelance class

The group $H\times K$ acts on $G:$ 
$$
    (h, k) : g \mapsto hgk^{-1}.
$$
And the orbits of this action are the double cosets $HgK.$

We can define a \Defn{double coset graph} as follows.
Let $G$ be a group, and $H$ a subgroup of $G.$
Choose $a\in G$ with $a^2\in H.$
We define a graph $\Gamma$
with vertices and edges as
\begin{align*}
    \mbox{V}(\Gamma) &= \{ Hg | g\in G\} \\
    \mbox{E}(\Gamma) &= \{ \{Hx, Hy\} | xy^{-1}\in G\}.
\end{align*}
Then $G$ acts on this graph by right multiplication.
This action is vertex-transitive with vertex stabiliser
$G_H = \{g \in G : Hg = H\} = H$ itself, which acts
transitively on the neighbours $Hah$ (for $h \in H$) of $H$.
Thus $\Gamma$ is arc-transitive \cite{Sabidussi1964}.


\cite{Kassel2010}
exercise 4.2.3: The Hecke algebra of $\GL_n(\Field_q).$


\bibliography{refs}{}
\bibliographystyle{abbrv}


\end{document}


