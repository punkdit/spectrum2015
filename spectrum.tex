
%\documentclass[12pt,notitlepage,aps,pra,longbibliography,nofootinbib,tightenlines]{revtex4}
\documentclass[12pt,notitlepage,longbibliography,nofootinbib,tightenlines]{revtex4}

%\documentclass[12pt,a4]{revtex4}
%\documentclass[12pt]{article}
%\documentclass[11pt, twocolumn]{article}

%\usepackage{epsf}
\usepackage{amsmath}
\usepackage{color}
\usepackage{natbib}
%\usepackage{cite}

\RequirePackage{amsmath}
\RequirePackage{amssymb}
\RequirePackage{amsthm}
%\RequirePackage{algorithmic}
%\RequirePackage{algorithm}
%\RequirePackage{theorem}
%\RequirePackage{eucal}
\RequirePackage{color}
\RequirePackage{url}
\RequirePackage{mdwlist}

\RequirePackage[all]{xy}
\CompileMatrices
\RequirePackage{hyperref}
\RequirePackage{graphicx}
%\RequirePackage[dvips]{geometry}

\def\Xcite#1{}

\begin{document}

%\title{Representations and Expansion in Gauge Code Hamiltonians}
\title{On the Spectrum of Gauge Code Hamiltonians}

\author{Simon Burton}
\affiliation{Centre for Engineered Quantum Systems, School of Physics, The University of Sydney}

\maketitle

\def\Complex{\mathbb{C}}
\def\Z{\mathbb{Z}}
\def\Ham{\mathcal{H}}
\def\Pauli{\mathcal{P}}
\def\Spec{\mbox{Spec}}
\def\Proveit{{\it (Proof??)}}
\def\GL{\mathrm{GL}}
\def\half{\frac{1}{2}}

%\section{Introduction}

This is an attempt to apply some ideas from
spectral graph theory to the study of Hamiltonians
built from (possibly non-commuting) Pauli algebra terms.

{\noindent\bf\underline{Motivating examples of gapless and gapped models.}}

The $XY$ model:
$$
    \Ham = \sum_i X_i X_{i+1} + Z_i Z_{i+1}.
$$
Transverse field Ising model:
$$
    \Ham = \sum_i X_i X_{i+1} + Z_i.
$$
In both these cases the existence of large weight stabilizer generators
causes the gapless behaviour.

As another example, take 
$$
    \Ham = \sum_i X_i + Z_i.
$$
In this case there are no stabilizers and indeed the Hamiltonian is gapped.

We need to be a bit careful what we mean by 
the weight of a stabilizer generator.
In this case, we define the weight of a stabilizer to mean the number
of terms in the Hamiltonian required to build (by multiplication)
a stabilizer. 
This rules out trivial cases such as 
$\Ham = \prod_{j=1}^{j=n} X_j,$ where the Hamiltonian is one big stabilizer
but is obviously gapped.

%\subsection{Motivation}

\section{Representations of gauge codes}

%It follows that the adjacency matrix $A$ of Cayley($G$, $S$)
%is the linear operator $A:\Complex[G]\to\Complex[G]$
%given by
%$$
%    A = \sum_{g\in G}\alpha(g) \rho_{\mathrm{reg}}(g)
%$$
%where $\rho_{\mathrm{reg}}$ is the left regular representation of
%$G$ on $\Complex[G],$
%and the indicator function $\alpha:G\to\Complex$ is
%$$
%    \alpha(g) := \left\{ \begin{array}{ll}
%1 &\mbox{if $g\in S$; and}\\
%0 &\mbox{otherwise.}\end{array} \right.
%$$
%
%Given a group representation $\rho:G\to \mathrm{GL}(V)$,
%and function $\phi:G\to \Complex,$
%the {\it Fourier transform} % of $\phi$ relative to $\rho$,
%$\hat\rho(\phi)\in \mathrm{GL}(V)$ is defined as
%$$\hat\rho(\phi) := \sum_{g\in G} \phi(g)\rho(g).$$
%
%Then we have
%$$
%    \hat\rho(\alpha) = \sum_{g\in S} \rho(g).
%$$
%
%\noindent{\bf Theorem.} 
%(See \cite{Kaski2002} and \cite{Diaconis1981})
%The spectrum of the adjacency matrix $A$ of a
%cayley graph Cayley($G$, $S$) equals the union
%of the spectra of fourier transforms of the indicator
%function $\alpha$ of the generating set $S$:
%$$
%    \Spec(A) = \bigcup_{\mathrm{irrep}\ \rho} \Spec(\hat\rho(\alpha))
%$$

\subsection{Cayley graphs}

\def\rhoreg{\rho_\mathrm{reg}}

The {\it Cayley graph} of a group $G$ and
generating set $S$ is denoted Cayley($G$, $S$).
It has nodes $G$ and edges $\{(g, sg) : g \in G, s \in S\}$.
We will always assume $S$ is closed under group inverse,
and so we can consider the Cayley graph as undirected.

The adjacency matrix of a graph with $N$ nodes is the 
$N$ by $N$ matrix $A$ with non-zero entries $A_{ij}=1$ corresponding
to edges $(i, j)$ of the graph.
It follows that the adjacency matrix $A$ of Cayley($G$, $S$)
is the linear operator $A:\Complex[G]\to\Complex[G]$
given by
$$
    A = \sum_{g\in S}\rho_{\mathrm{reg}}(g)
$$
where $\rho_{\mathrm{reg}}$ is the left regular representation of
$G$ on $\Complex[G].$

Using representation theory we have that 
$\rhoreg:G\to \Complex[G]$
decomposes as the dirrect sum of irreducible
representations $\rho_k:G\to \GL(V_k)$
and so
$$
    A = \sum_{g\in S}\bigoplus_k \rho_k(g)
$$

block diagonalizes $A$,
with each $\rho_k$ possibly appearing multiple
times in the direct sum over $k$.
See \cite{Kaski2002} and \cite{Diaconis1981} for more details.

\subsection{Gauge codes}

The Pauli group $\Pauli_1$ is normally 
defined as a set of matrices closed under
matrix multiplication, but we can define
it abstractly as the group generated
by the (abstract) elements $\{m, X, Z\}$ with
relations as follows:
$$
m^2=I,\ X^2=I,\ Z^2=I,\ mXmX=I,\ mZmZ=I,\ \mbox{and}\  mZXZX=I,
$$
where $I$ is the group idenity.
Actually, $m$ is generated by $X$ and $Z$, so
it is not necessary to include $m$ in the generating set,
but here it simplifies the relations.

To define the $n$-qubit Pauli group $\Pauli_n$, 
we use the $2n+1$ element 
generating set $\{m, X_1, .., X_n, Z_1, .., Z_n\}$
with relation $m^2=I$ as before, and
$$
\begin{array}{c}
X_i^2=I,\ Z_i^2=I,\ mX_imX_i=I,\ mZ_imZ_i=I,\ mZ_iX_iZ_iX_i=I, 
\mbox{\ for\ } i=1,...n,\\
Z_iX_jZ_iX_j=I, \mbox{\ for\ } i, j = 1,..,n,\ i\ne j.
\end{array}
$$

Note that $m$ commutes with all elements of $\Pauli_n$
and squares to the idenity, so we will denote this
element as $-1.$ Similarly, $\pm 1$ is thought of as the
set $\{m, I\},$ and $-X$ is $mX$, etc.

%Elements of $\Pauli_n$ that consist of
%products of only $I$ or $X$ will
%be call $X$-type elements (or operators)
%and similarly $Z$-type elements are 
%products of only $I$ or $Z$.
The subgroup of $\Pauli_n$ generated by
the elements $\{X_1,...,X_n\}$ % $\{X_i\}_{i=1,..,n}$ 
is denoted $\Pauli_n^X.$ These are the $X$-type
elements. Similarly,
 $\{Z_1,...,Z_n\}$ generates % $\{Z_i\}_{i=1,..,n}$ 
the subgroup of $Z$-type elements $\Pauli_n^Z$.

Every element $g\in\Pauli_n$ can be written
uniquely as a product $g = \pm g_X g_Z,$
where $g_X$ is an $X$-type operator and $g_Z$
is a $Z$-type operator.
This gives the size of the group as:
$$
    |\Pauli_n| = 2^{2n+1}.
$$

We now define the
Pauli {\it representation} 
of the Pauli group as a group homomorphism:
$$
    \rho_{\mathrm{pauli}} : \Pauli_n \to \GL(V)
$$
where $V$ is the $2^n$ dimensional state space of $n$ qubits.
On the independant generators 
$\{X_1, .., X_n, Z_1, .., Z_n\},\ \rho_{\mathrm{pauli}}$
is defined as the following tensor product of $2\times 2$ matrices:
$$
\rho_{\mathrm{pauli}}(X_i) := \bigotimes_{j=1}^n \left\{ \begin{array}{ll}
\left( \begin{array}{ll}
1&0\\
0&1\end{array} \right) &\mbox{for $j\ne i$,}\\
\\
\left( \begin{array}{ll}
0&1\\
1&0\end{array} \right) &\mbox{for $j=i$} \end{array}
\right\},\ 
\rho_{\mathrm{pauli}}(Z_i) := \bigotimes_{j=1}^n \left\{ \begin{array}{ll}
\left( \begin{array}{ll}
1&0\\
0&1\end{array} \right) &\mbox{for $j\ne i$,}\\
\\
\left( \begin{array}{rr}
1&0\\
0&-1\end{array} \right) &\mbox{for $j=i$}\end{array}
\right\}.
$$

%on $n$ qubits, $\Pauli_n,$
%as the set of $n$-fold tensor products
%of the matrices $\pm I, X, Z:$
%$$
%I = \left( \begin{array}{ll}
%1&0\\
%0&1\end{array} \right),\quad
%X = \left( \begin{array}{ll}
%0&1\\
%1&0\end{array} \right),\quad
%Z = \left( \begin{array}{ll}
%1&0\\
%0&-1\end{array} \right).
%$$

Normally the image of 
$\rho_{\mathrm{pauli}}$ is thought of as the
Pauli group itself, and we are indeed free to think
that way because $\rho_{\mathrm{pauli}}$ is a group
isomorphism.
It turns out that $\rho_{\mathrm{pauli}}$ is an
irreducible representation ({\it irrep}) of $\Pauli_n$ and the
only other irreps of $\Pauli_n$ are 
the $1$-dimensional irreps $\rho:\Pauli_n\to\Complex$
defined on the independant generators as:
    $$ \rho(X_i) = \pm 1,\quad \rho(Z_i) = \pm 1.$$

So we have $2^{2n}$ many $1$-dimensional irreps,
and a single $2^n$-dimensional irrep.
Summing the squares of the dimensions
shows that we have a complete set of irreps of $\Pauli_n.$

We now define a {\it gauge} subgroup $G$ of $\Pauli_n$
by choosing a set of generators $S\subset \Pauli_n,$
%for some subgroup $G$ of $\Pauli_n:$
$$ G := \langle S\rangle.$$
We will assume $G$ is not abelian, which is
equivalent to the condition that $-I\in G.$
We also restrict $S$ to only contain Hermitian operators,
which is equivalent to requiring that $g^2=I$ for all $g\in S.$
Now let $H$ be the largest subgroup of $G$ not containing
$-I.$
$H$ is then an abelian subgroup,
also known as the {\it stabilizer} subgroup.
%(Note that each of the stabilizers commutes with the Hamiltonian.)
$G$ decomposes as a direct product:
$$G = H\times R,$$
where $R\cong P_r$ for some $1\le r\le n,$
and $H\cong \Z_2^{m}$ for $0\le m<n.$
Therefore, $|G| = |H| |R| = 2^{m+2r+1}.$
We call $R$ the {\it reduced} gauge group.
We consider both $H$ and $R$ to be subgroups of $G.$
Let $\phi:P_r\to R$ be a group isomorphism,
%then $R_0 := \{\phi(X_i), \phi(Z_i)\}_{1\le i\le r}$
then $R_0 := \{\phi(X_i), \phi(Z_i)\}_{i=1,..,r}$
is a set of independant generators of $R.$
We also let $H_0$ be a set of $m$ independant generators of $H.$

The $1$-dimensional irreps $\rho:G\to \Complex,$
are now defined by
specifying the action of $\rho$ on the independant generators:
$$
    \rho(h)=\pm 1\ \mbox{for}\ h\in H_0,
    \quad \rho(\phi(X_i)) = \pm 1,\quad \rho(\phi(Z_i)) = \pm 1.
$$

This gives all $2^{m+2r}$ of the $1$-dimensional irreps.
Finally, there are $2^m$ many $2^r$-dimensional irreps given by:
$$
    \rho(h)=\pm I^{\otimes 2^r}\ \mbox{for}\ h\in H_0,
    \quad \rho(\phi(X_i)) = X_i,\quad \rho(\phi(Z_i)) = Z_i.
$$


The Hamiltonian of interest is normally defined
as the negative sum of terms from $S,$ but here
we will reverse the sign:
$$ \Ham := \sum_{g\in S} \rho_{\mathrm{pauli}}(g).$$

The goal is to decompose $\Ham$ into blocks as
$$
    \Ham = \bigoplus_{\mathrm{irrep}\rho} \sum_{g\in S}\ \rho(g),
$$
and we will notate each block as
$\Ham_\rho := \sum_{g\in S}\rho(g)$
for each irrep $\rho$ appearing in $\Ham.$

The form of $\Ham$ is seen to be very similar
to the adjacency matrix of Cayley$(G, S)$ but
instead of the regular representation we are
using the Pauli representation.
We use the following map to
relate these two representations:
%transfer information about $A$ over to $\Ham$.
%The Hamiltonian is related to the adjacency matrix
%$A$ of Cayley$(G, S)$ by the following commutation relation.

\noindent{\bf Theorem.}
Define a linear map $f:\Complex[G]\to\Complex[G_X]$ as follows.
Any element $g\in \Pauli_n$ can be written as $g = \pm g_x g_z$
and we set $f(g) = f(\pm g_x g_z) := \pm g_x.$
Then,
$$
    f A = \Ham f.
$$

\noindent{\bf Proof.} \Proveit ...
\qed

It is easy to see that any eigenvector $v$ of $A$
with eigenvalue $\lambda$
is either in the kernel of $f$, or otherwise $fv$
is an eigenvector of $\Ham$ with eigenvalue $\lambda.$
Furthermore, $f$ is full-rank \Proveit, and so {\it all} the eigenvectors
of $\Ham$ are of the form $fv$ for some eigenvector $v$ of $A.$

We write the (distinct) eigenvalues of $\Ham$ in decreasing
order:
$$ \lambda_1 > \lambda_2 > ... $$
Of particular interest is the gap between the first
and second eigenvalues,
$\epsilon := \lambda_1 - \lambda_2.$

There is a well understood theory of
expansion in Cayley graphs that shows how the
structure of the group $G$ leads to
gapped behaviour of $A.$
Unfortunately, the top eigenvector of $A$ is
in the kernel of $f$ and so these results
do not help us show gapped behaviour of $\Ham.$

The above commutation relation
for $f$ is the definition of an {\it intertwining} map,
and it is a general result that such maps either 
preserve an irreducible representation or send them to zero.

\noindent{\bf Theorem.}
The (images of) all the one-dimensional irreps are contained in
the kernel of $f$. All the other irreps are preserved.

\noindent{\bf Proof.} \Proveit ...
\qed

In the sequal we will make the identification
between $g$ and $\rho_{pauli}(g)$.
So terms such as $Z$ and $X$ are understood
to be the corresponding Pauli linear operators.

\subsection{Example: 2D compass model}

Here we consider the two dimensional compass model.
We coordinatize the qubits on a square 
lattice of\ $l\times l$\ sites,
$(i, j)$\ for\ $1\le i, j\le l.$
This gives $n = l^2.$
For the single qubit Pauli operators acting on site
$(i, j)$ we coordinatize with subscripts $ij$, 
with $i$ and $j$ understood modulo $l$.
The generators of the gauge group are
$$
    S = \big\{ X_{ij}X_{i,j+1},\ Z_{ij}Z_{i+1,j}\ \mbox{for}\ 1\le i, j\le l\big\}.
$$
We write generators of the reduced
gauge group in anti-commuting pairs:
$$
    R_0 = \big\{ X_{i1}X_{ij},\ Z_{1j}Z_{ij}\ \mbox{for}\ 2\le i, j\le l\big\}.
$$
This makes it clear the isomorphism $\phi : R_0 \to \Pauli_r$ to use,
and we again use pairs $i,j$ to coordinatize $\Pauli_r$:
$$
    \phi(X_{i1}X_{ij}) = X_{i-1,j-1}, \ \ \phi(Z_{1j}Z_{ij}) = Z_{i-1,j-1},\ \mbox{for}\ 2\le i, j\le l.
$$
The generators for the stabilizers are
$$
    H_0 = \big\{ \prod_{i=1}^l X_{ij}X_{i,j+1},\ \prod_{i=1}^l Z_{ji}Z_{j+1,i}\ \mbox{for}\ 1\le j\le l-1\big\}.
$$
The logical operators are generatred by $L_0 = \big\{ \prod_i X_{i1}, \prod_j Z_{1j} \}.$
These sets have cardinalities:
$$|S|=2l^2,\ |R_0| = 2(l-1)^2,\ |H_0| = 2(l-1).$$
And we note that $\frac{1}{2}|L_0| + |H_0| + \frac{1}{2}|R_0| = n.$
%Now we can define the irreps of $G$.
Now we write down the values of the
irreps on the gauge operators.
Note the transposition symmetry between the $X$ and $Z$-type operators:
\begin{align*}
\rho(X_{i1} X_{i2}) &= X_{i-1,1} &
\rho(Z_{1i} Z_{2i}) &= Z_{1,i-1} &\mbox{for}\ 2\le i\le l\\
\rho(X_{ij} X_{i,j+1}) &= X_{i-1,j-1} X_{i-1,j} &
\rho(Z_{ji} Z_{j,i+1}) &= Z_{j-1,i-1}Z_{j,i-1} &\mbox{for}\ 2\le i, j\le l\\
\rho(X_{1j} X_{1,j+1}) &= \pm \prod_{i=1}^l X_{i,j-1} X_{ij} &
\rho(Z_{j1} Z_{j+1,1}) &= \pm \prod_{i=1}^l Z_{j-1,i} Z_{ji} &\mbox{for}\ 2\le j<l\\
\rho(X_{11} X_{12}) &= \pm \prod_{i=1}^l X_{i1} &
\rho(Z_{11} Z_{21}) &= \pm \prod_{i=1}^l Z_{1i}.
\end{align*}
We sum all these terms to find 
the form of the hamiltonian in each block:
$$
\Ham_\rho = \sum_{g\in S} \rho(g) = \sum_{1\le i,j<l} \rho(X_{ij}X_{i,j+1}) + \rho(Z_{ij}Z_{i+1,j}).
$$
We note that in \cite{Brzezicki2013}, they perform a
(ad-hoc?) transformation of the compass model
which results in a similar $(l-1)\times(l-1)$ lattice
of spins, but they have more terms in their Hamiltonian.
It is not clear if these modifications are essential to
their analysis.

\subsection{Example: Kitaev honeycomb model}

The Kitaev honeycomb model is build from spins on
the sites of a hexagonal lattice. 
The lattice of linear size $l$ has $n=2l^2$ sites
which we coordinatize using integer triples $i, j, k$
with $1\le j, k\le l$ and $k=1, 2.$
We use periodic boundary conditions so $i, j$ are
to be taken modulo $l$.
The edges of the lattice are in one-to-one
correspondence with the generators $S$:
$$
S = \big\{X_{ij1}X_{ij2},\ Z_{ij2}Z_{i+1,j1},\ Y_{ij1}Y_{i-1,j+1,2}
\ \mbox{for}\ 1\le i,j\le l\big\}.
$$

Stabilizers are generated from closed strings of
gauge operators. This gives independant stabilizer generators
from each hexagon, less one, as well as two
homologically non-trivial loops.
The number of hexagons is $\frac{1}{2}n$ and
so we find $|H_0|=\half n+1.$
There are no logical operators, so we
must have $|R_0|=n-2.$

In \cite{Kells2009} they introduce a set of
mutually anti-commuting string operators $K_0.$
With periodic boundary conditions $K_0$ forms an
independant generating set of $R$ of size $n-2.$
To construct an isomorphism $\phi: R\to \Pauli_r$
we map elements of $K_0$ to the following independant generating
set of $\Pauli_r$:
$$
%\big\{\prod_{i=1}^{j-1} Z_i X_j,\ \prod_{i=1}^{j-1} Z_i Y_j\ \mbox{for}\ 1\le j\le r\big\}.
\big\{c_{2j}:=Z_1...Z_{j-1} X_j,\ c_{2j+1}:=Z_1...Z_{j-1} Y_j\ \mbox{for}\ 1\le j\le r\big\}.
$$

{\it XXX Prove this is an isomorphism }

The $c_j$ are now paired Majorana fermion operators, 
and each block in the Hamiltonian
is seen to be quadratic in these operators:
$$
    \Ham_\rho = \sum_{ij} \Gamma_{ij}(\rho) c_i c_j
$$
where the coefficients $\Gamma_{ij}$
are dependant on the irrep $\rho.$

%A similar analysis applies to the ising model, Jordan-Wigner... 
%http://arxiv.org/pdf/1504.01444 page 86.

\section{Bounding the gap}

In this section we view the Hamiltonian as the adjacency matrix of
a {\it weighted} graph.
We restrict our attention to Hamiltonians whose off-diagonal entries
are non-negative.
This can be achieved by considering Hamiltonians where each term
involves either $X$-type operators or $Z$-type operators but not both.
That is, $S$ consists only of $X$-type operators and $Z$-type operators.
We also shift the Hamiltonian by a constant energy, so that
the diagonal entries are non-negative:
$$
%\Ham := \sum_{g\in S} \rho_{\mathrm{pauli}}(g).
\Ham := \sum_{g\in S} \rho_{\mathrm{pauli}}(g) + kI.
$$
This does not change
the spectral gap or eigenvalues.

A simple variational argument
shows that the top eigenvector (the {\it groundstate})
can be chosen to have all positive entries
(this is the Perron-Frobenius theorem)
and therefore is stabilized:

\noindent{\bf Theorem.}
Every groundstate is stabilized.

\noindent{\bf Proof.}
\qed


In \cite{Friedland2002}, they derive the following cheeger inequality
by considering bi-partitions of the graph. We will do the
same, but using matrix block notation.

Let $v_2$ be a second eigenvector, $ \Ham v_2 = \lambda_2 v_2 $ 
and $||v_2||=1$.
We bi-partition the space 
so that $v_2$ has (vector) blocks:
$$
v_2 = \left( \begin{array}{l}
x\\
y\end{array} \right)\quad
$$
with $x\ge 0$ and $y\le 0,$ component-wise.
Let the blocks of $\Ham$ under the same partition be:
$$
\Ham = \left( \begin{array}{ll}
A&C\\
C^\top&B\end{array} \right).\quad
$$
If we denote $\lambda_1(A)$ as the top eigenvalue of $A$ and
$\lambda_1(B)$ as the top eigenvalue of $B$,
then
\begin{align*}
\lambda_2 = v_2^\top \Ham v_2 &= x^\top A x + 2 x^\top C y + y^\top B y \\
        &\le x^\top A x + y^\top B y\ \le\ ||x||^2 \lambda_1(A) + ||y||^2 \lambda_1(B) \\
        &\le \mbox{min}(\lambda_1(A), \lambda_1(B))\ \le\ \lambda_1.
\end{align*}

Defining the following constant as a maximisation over
all bi-partitions of $\Ham:$
$$
    \nu(\Ham) := \max_{A, B}\ \mbox{min}((\lambda_1(A), \lambda_1(B))
$$
the above calculation shows that
$$
    \lambda_2 \le \nu(\Ham) \le \lambda_1.
$$

To bound $\lambda_2$ from below, we use a variational argument.
For any unit vector $v$ orthogonal to the top eigenspace of $\Ham$ we
have $v^\top \Ham v \le \lambda_2.$


See also,
\cite{AlShimary2010}.
And \cite{Jarret2015}.

%
%\begin{center}\includegraphics[width=0.3\textwidth]{pic-braid-group.pdf}\end{center}

%\newpage

\appendix

\section{Gap source code}

This is source code for the GAP system \cite{GAP4} to display
the irreducible representations of $\Pauli_2.$

\begin{verbatim}
F := FreeGroup("m", "xi", "zi", "ix", "iz");;
m := F.1;; xi := F.2;; zi := F.3;; ix := F.4;; iz := F.5;;
G := F / [ix*xi*ix*xi, m*iz*ix*iz*ix, m*zi*m*zi, xi*xi, xi*m*xi*m, 
    zi*iz*zi*iz, m*xi*zi*xi*zi, ix*m*ix*m, m*xi*m*xi, xi*ix*xi*ix, 
    iz*iz, ix*zi*ix*zi, m*m, zi*m*zi*m, ix*ix, 
    zi*ix*zi*ix, xi*iz*xi*iz, m*iz*m*iz, m*ix*m*ix, m*zi*xi*zi*xi, 
    iz*zi*iz*zi, m*ix*iz*ix*iz, iz*m*iz*m, iz*xi*iz*xi, zi*zi];;
Print(Order(G));
LoadPackage( "repsn" );;
chi := Irr(G);
Print(chi);
for c in chi do
    rep := IrreducibleAffordingRepresentation(c);
    Print(rep);
    Print("\n");
od;
\end{verbatim}


\bibliography{refs}{}
\bibliographystyle{abbrv}


\end{document}


